\documentclass[12pt,fleqn,answers]{exam}
\usepackage{pifont}
\usepackage{dingbat}
\usepackage{amsmath,amssymb}
\usepackage{epsfig}
\usepackage[colorlinks=true,linkcolor=black,anchorcolor=black,citecolor=black,filecolor=black,menucolor=black,runcolor=black,urlcolor=black]{hyperref}
\usepackage[letterpaper, margin=0.75in]{geometry}
\addpoints
\boxedpoints
\pointsinmargin
\pointname{pts}

\usepackage[final]{microtype}
\usepackage[american]{babel}
%\usepackage[T1]{fontenc}
\usepackage{fourier}
\usepackage{isomath}
\usepackage{upgreek,amsmath}
\usepackage{amssymb}

\newcommand{\dotprod}{\, {\scriptzcriptztyle
    \stackrel{\bullet}{{}}}\,}

\newcommand{\reals}{\mathbf{R}}
\newcommand{\lub}{\mathrm{lub}} 
\newcommand{\glb}{\mathrm{glb}} 
\newcommand{\complex}{\mathbf{C}}
\newcommand{\dom}{\mbox{dom}}
\newcommand{\range}{\mbox{range}}
\newcommand{\cover}{{\mathcal C}}
\newcommand{\integers}{\mathbf{Z}}
\newcommand{\vi}{\, \mathbf{i}}
\newcommand{\vj}{\, \mathbf{j}}
\newcommand{\vk}{\, \mathbf{k}}
\newcommand{\bi}{\, \mathbf{i}}
\newcommand{\bj}{\, \mathbf{j}}
\newcommand{\bk}{\, \mathbf{k}}
\newcommand{\dist}{\, \mathrm{dist}}
\DeclareMathOperator{\Arg}{\mathrm{Arg}}
\DeclareMathOperator{\Ln}{\mathrm{Ln}}
\newcommand{\imag}{\, \mathrm{i}}

\usepackage{xcolor}
\shadedsolutions
\definecolor{SolutionColor}{rgb}{0.95,0.95,0.95}

\usepackage{graphicx}
\newcommand\AM{{\sc am}}
\newcommand\PM{{\sc pm}}
     
\usepackage{twemojis}
\newcommand{\quiz}{3}
\newcommand{\term}{Spring}
\newcommand{\due}{9:55 \AM}
\newcommand{\class}{MATH 102}
\begin{document}
\large
\vspace{0.1in}
\noindent\makebox[3.0truein][l]{\textbf{\class, \term \/ \the\year}}
\textbf{Name:} \hrulefill \\
\noindent \makebox[3.0truein][l]{\textbf{In class work \quiz}}
\textbf{Row and Seat}:\hrulefill\\
\vspace{0.1in}


\noindent  In class work  \quiz\/  has questions 1 through  \numquestions \/ with a total of  \numpoints\/  points.   
 This assignment is due at the end of the class period (\due).

\vspace{0.1in}


\begin{questions} 

\question [5] Find the \emph{center} and \emph{radius} of the circle
\begin{equation*}
    {{x}^{2}}-2 x+{{y}^{2}}+2 y=4.
\end{equation*}
Do this by matching to the general equation of a circle centered at 
$(x=h,y=k)$ and radius $r$
\begin{equation*}
    x^2 -2 h x + h^2 + y^2 - 2 k y + k^2 = r^2.
\end{equation*}

\begin{solution}%[3.5in]
    \begin{align*}
        \left[x^2 - 2 x + y^2 + 2 y  =  4 \right ]
          &= \left[ x^2 - 2 x  + \boxed{1} + y^2 + 2 y + \boxed{1}  =  4 + \boxed{1} + \boxed{1} \right] \\
           &= \left[ x^2 - 2 x  + \boxed{1} + y^2 + 2 y + \boxed{1}  =  6 \right] \\
    \end{align*}
    We need to match
    \begin{align*}
         - 2x  &= -2 h x, \\
          2y   &= -2 k y,\\
          6     &= r^2.
    \end{align*}
    Solving for $h,k$ and $r$ gives $h=1, k=-1$, and $r =\sqrt{6}$. So the center is \mbox{$(x=-1, y=1)$} and radius the radius is 
    $r = \sqrt{6}$.
\end{solution}


\question [5] The number of lawns $L$ a work crew can mow in a day 
varies jointly with the number of people $N$ in the crew and with
the time $T$  they work in a day.

Given that $L = 8$ when $N = 5$ and $T = 6$, find $L$ when
$N = 8$ and $T = 10$.
\begin{solution}[3.5in]
There is a constant $k$ such that $L =k N T$. This formula should make sense--doubling the size of the work crew, we
should be able to mow twice as many lawns--the formula shows that is true. The same is true for working twice as long.

Pasting in $L = 8$ when $N = 5$ and $T = 6$ into $L =k N T$ yields $8 = 30 k$.  So $k = \frac{4}{15}$.  That makes
our formula
\begin{equation*}
L = \frac{4}{15} N T.
\end{equation*}
Pasting in $N = 8$ and $T = 10$ gives
\begin{equation*}
L = \frac{4}{15} \times 80 = \frac{64}{3} \approx 21.3.
\end{equation*}




\end{solution}


\question The corpulence index CI is an alternative to the 
body mass index BMI. The CI thought to be more based on physiology than 
is the BMI. The CI varies jointly with the weight $w$ and with the inverse 
\emph{cube} of the height $L$. Thus for some number $k$, we have
\begin{equation*}
    \mathrm{CI} = k  \frac{w}{L^3}.
\end{equation*}

\begin{parts}

  

    \part [5]  Usan Bolt, a world record holder for the 100 meter dash, is 77 inches
     tall and weighs 207 pounds. Given that Usan Bolt's 
    CI is 12.4, find the numerical value of the  proportionality constant $k$.
    \begin{solution}[3.5in] Pasting in the data gives
     \begin{equation*}
          12.4 = k \times  \frac{207}{77^3}
     \end{equation*}
     Solving this for $k$ gives $k \approx 27,348$.
     
    \end{solution}
    \part Florence Griffith Joyner, a world record holder for the 100 meter dash, 
    is 67 inches tall and weighs 126 pounds.  Find the CI of Florence Griffith Joyner.
    \begin{solution}%[3.5in]
    We have
    \begin{equation*}
          CI = 27, 348  \times \frac{126}{67^3} \approx 11.5.
     \end{equation*}
     
     \textbf{Teacher's Embellishment} The BMI is widely used to determine if a person has an appropriate weight, but
     the logic behind the BMI is silly.  The CI is somewhat better based on fact than the BMI.   Assume a human is
     a right circular cylinder with radius $r$ and height $h$.  The density $\rho$ of this person is
     \begin{equatiomn
    \end{solution}
    
   % \vfill

\end{parts}

\end{questions}
\end{document}