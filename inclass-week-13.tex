\documentclass[12pt,fleqn,answers]{exam}
\usepackage{pifont}
\usepackage{dingbat}
\usepackage{amsmath,amssymb}
\usepackage{epsfig}
\usepackage[colorlinks=true,linkcolor=black,anchorcolor=black,citecolor=black,filecolor=black,menucolor=black,runcolor=black,urlcolor=black]{hyperref}
\usepackage[letterpaper, margin=0.75in]{geometry}
\usepackage{tikz}
\usetikzlibrary{arrows}
\addpoints
\boxedpoints
\pointsinmargin
\pointname{pts}

\usepackage{pdfpages}
\usepackage[final]{microtype}
\usepackage[american]{babel}
\usepackage[T1]{fontenc}
\usepackage{fourier}
%\usepackage{eucal}

\usepackage{isomath}
\usepackage{upgreek,amsmath}
\usepackage{amssymb}
\usepackage{siunitx}

\newcommand{\dotprod}{\, {\scriptzcriptztyle
    \stackrel{\bullet}{{}}}\,}

\newcommand{\reals}{\mathbf{R}}
\newcommand{\lub}{\mathrm{lub}} 
\newcommand{\glb}{\mathrm{glb}} 
\newcommand{\complex}{\mathbf{C}}
\newcommand{\dom}{\mbox{dom}}
\newcommand{\range}{\mbox{range}}
\newcommand{\cover}{{\mathcal C}}
\newcommand{\integers}{\mathbf{Z}}
\newcommand{\degree}{\mathrm{degree}}
\newcommand{\vi}{\, \mathbf{i}}
\newcommand{\vj}{\, \mathbf{j}}
\newcommand{\vk}{\, \mathbf{k}}
\newcommand{\bi}{\, \mathbf{i}}
\newcommand{\bj}{\, \mathbf{j}}
\newcommand{\bk}{\, \mathbf{k}}
\newcommand{\dist}{\, \mathrm{dist}}
\DeclareMathOperator{\Arg}{\mathrm{Arg}}
\DeclareMathOperator{\Ln}{\mathrm{Ln}}
\newcommand{\imag}{\, \mathrm{i}}

\usepackage{tikz}
\usepackage{amsmath}
\usetikzlibrary{arrows}
\usepackage{xcolor}
\shadedsolutions
\definecolor{SolutionColor}{rgb}{0.95,0.95,0.95}

\usepackage{graphicx}
\newcommand\AM{{\sc am}}
\newcommand\PM{{\sc pm}}
     
%\usepackage{twemojis}
\newcommand{\quiz}{12}
\newcommand{\term}{Spring}
\newcommand{\due}{9:55 \AM}
\newcommand{\class}{MATH 102}
\begin{document}
\large
\vspace{0.1in}
\noindent\makebox[3.0truein][l]{\textbf{\class, \term \/ \the\year}}
\textbf{Name:} \hrulefill \\
\noindent \makebox[3.0truein][l]{\textbf{In class work \quiz}}
\textbf{Row and Seat}:\hrulefill\\
%\vspace{0.1in}

\small
\begin{flushleft}
  \emph{
    “Money buys everything except love, personality, freedom, immortality, silence, peace.”}
    \hfill \sc{Carl Sandburg}
 \end{flushleft}
 \large
\noindent  In class work  \quiz\/  has questions 1 through  \numquestions \/ with a total of  \numpoints\/  points.   
 This assignment is due at the end of the class period (\due).
This assignment is printed on \textbf{both} sides of the paper.
\vspace{0.1in}


\begin{questions} 

\question [2] Andy is saving his money to purchase a \$12,495 Martin D--18 guitar.
To save for the guitar, he invests \$10,000 into a bank CD with an APY of 5.0\%. 
How long will Andy need to wait until he can purchase the Martin D--18?

\begin{solution}[3.0in]
    We need to solve $12495 = 1000 \times 1.05^T$ for $T$. Since the unknown
    $T$ is an exponent, we'll use the \emph{logarithm trick}. Specifically,
    \begin{align*}
      \left[12495 = 10000 \times 1.05^T \right] &= \left[1.2495 = 1.05^T \right], &\mbox{(divide by 10000)} \\
                     &= \left[\log_{10}(1.2495) = \log_{10}(1.05^T) \right],  &\mbox{(logarithm trick)} \\
                     &= \left[\log_{10}(1.2495) = T \log_{10}(1.05) \right],  &\mbox{(logarithm property)} \\
                     &= \left[ T = \frac{\log_{10}(1.2495)}{\log_{10}(1.05)} \right],  &\mbox{(divide by $\log_{10}(1.05)$)} \\
                     &= \left[ T \approx 4.6 \mbox{ years} \right].  &\mbox{(arithmetic)}
    \end{align*}
Remember: $\log_{10}$ is \textbf{not} a multiplicative factor 
in the quotient $\displaystyle \frac{\log_{10}(1.2495)}{\log_{10}(1.05)}$. 
So do \textbf{not} cancel $\log_{10}$ in this expression.


\textbf{FYI:} The actor, storyteller, and musician Andy Griffith played a Martin D-18 guitar in the movie \emph{A face in the Crowd}\
that was painted black. He had the guitar restored and played it professionally
for many years. 

\quad Elvis owned two Martin D--18 guitars. His 1942 model recently sold for \$1.32 million. For this problem, I looked on eBay for the cost of
a Martin D--18 that wasn't owned by a celebrity. 
\end{solution}

\question [2] In April 2004, Ms Oro purchases one ounce of gold for \$647. 
Today (that is, twenty years later), Ms Oro sells her gold for \$1,995.
Find the APY for this investment.
\begin{solution}[3.0in]
We need to solve $1995 = 647 \times (1+r)^{20}$ for $r$. Since 
the unknown is in the base of an exponential, we use the \emph{root trick}.
Specifically,
\begin{align*}
    \left[1995 = 647 \times (1+r)^{20} \right] &= \left[\frac{1995}{647} = (1+r)^{20} \right], & \mbox{(divide by 647)}\\
               &= \left[\left(\frac{1995}{647}\right)^{1/20} = (1+r) \right], & \mbox{(root trick)} \\
               &= \left[r = \left(\frac{1995}{647}\right)^{1/20} - 1 \right], & \mbox{(subtract 1)}\\
               &= \left[ r \approx 5.79\% \right]. &\mbox{(arithmetic)}
\end{align*}
You might be more comfortable doing the arithmetic as you go, instead
of saving it up until the end. That's okay, but I think it makes it
more difficult to check the work, increases chances of miscopying
numbers, and it potentially decreases accuracy (by rounding intermediate results).
Try saving the arithmetic to the end--you might like it.

\textbf{FYI:} By investing in a stock index fund, Ms Oro would have
gotten a higher APY on her investment. For the cost of gold, I used 
actual historical data--I didn't just make it up.

And by the way: the Spanish word for gold is `oro.'


\end{solution}


\question To save for the purchase of a new Aston Martin DB11,
James purchases a bond with a value of \$100,000 when it matures
in 30 years.

\begin{parts}

    \part [2] At the time of purchase, the 30 year APY is 5.0\%.
    Find the \emph{purchase price} of the bond.
    \begin{solution}[3.0in] We need to solve 
        $100000 = P \times 1.05^{30}$ for $P$. We have
        \begin{equation}
             P = \frac{100000}{1.05^{30}} \approx 23137.74.
        \end{equation}

    \end{solution}

    \part [2] After five years, James decides to sell his bond and
    to use the proceeds to purchase an organic cacao farm in Hawaii.
    At the time of sale, the 30 year APY is 4.0\%. Find the 
    sale price of the bond.
    \begin{solution}[3.0in] The future value of the bond is still
        \$100,000. Think of it this way: if you purchase the bond from 
        James and hold onto it until maturity, the Federal government
        will pay you \$100,000. So its future value is still \$100,000.        
        What has changed is both the time to maturity (was
        30 years, now is 25) and the APY (was 5.0\%, not is 4.0\%).
        We now need to solve $100000 = P \times 1.04^{25}$ for $P$. 
        We have
        \begin{equation}
             P = \frac{100000}{1.04^{25}} \approx 37511.68.
        \end{equation}

    \quad James made a pretty good investment. He purchased the bond for 
    \$23,137.74 and sold it five years later for \$37,511.68, making
    a profit of \$14, 373.94. The APY of this investment can be found by solving
    $37511.68 = 23137.74 \times (1+r)^5$ for $r$. The solution 
    is $r = 10.1\%$. And that's much greater than the APY of
    the bond.
    
    
    \emph{Falling interest rates make bonds more 
    valuable; rising interest rates make bonds less valuable.}


    \end{solution}




\end{parts}

\end{questions}
\end{document}
