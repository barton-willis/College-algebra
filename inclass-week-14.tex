\documentclass[12pt,fleqn,answers]{exam}
\usepackage{pifont}
\usepackage{dingbat}
\usepackage{amsmath,amssymb}
\usepackage{epsfig}
\usepackage[colorlinks=true,linkcolor=black,anchorcolor=black,citecolor=black,filecolor=black,menucolor=black,runcolor=black,urlcolor=black]{hyperref}
\usepackage[letterpaper, margin=0.75in]{geometry}
\usepackage{tikz}
\usetikzlibrary{arrows}
\addpoints
\boxedpoints
\pointsinmargin
\pointname{pts}

\usepackage{pdfpages}
\usepackage[final]{microtype}
\usepackage[american]{babel}
\usepackage[T1]{fontenc}
\usepackage{fourier}
%\usepackage{eucal}

\usepackage{isomath}
\usepackage{upgreek,amsmath}
\usepackage{amssymb}
\usepackage{siunitx}

\newcommand{\dotprod}{\, {\scriptzcriptztyle
    \stackrel{\bullet}{{}}}\,}

\newcommand{\reals}{\mathbf{R}}
\newcommand{\lub}{\mathrm{lub}} 
\newcommand{\glb}{\mathrm{glb}} 
\newcommand{\complex}{\mathbf{C}}
\newcommand{\dom}{\mbox{dom}}
\newcommand{\range}{\mbox{range}}
\newcommand{\cover}{{\mathcal C}}
\newcommand{\integers}{\mathbf{Z}}
\newcommand{\degree}{\mathrm{degree}}
\newcommand{\vi}{\, \mathbf{i}}
\newcommand{\vj}{\, \mathbf{j}}
\newcommand{\vk}{\, \mathbf{k}}
\newcommand{\bi}{\, \mathbf{i}}
\newcommand{\bj}{\, \mathbf{j}}
\newcommand{\bk}{\, \mathbf{k}}
\newcommand{\dist}{\, \mathrm{dist}}
\DeclareMathOperator{\Arg}{\mathrm{Arg}}
\DeclareMathOperator{\Ln}{\mathrm{Ln}}
\newcommand{\imag}{\, \mathrm{i}}

\usepackage{tikz}
\usepackage{amsmath}
\usetikzlibrary{arrows}
\usepackage{xcolor}
\shadedsolutions
\definecolor{SolutionColor}{rgb}{0.95,0.95,0.95}

\usepackage{graphicx}
\newcommand\AM{{\sc am}}
\newcommand\PM{{\sc pm}}
     
%\usepackage{twemojis}
\newcommand{\quiz}{14}
\newcommand{\term}{Spring}
\newcommand{\due}{9:55 \AM}
\newcommand{\class}{MATH 102}
\begin{document}
\large
\vspace{0.1in}
\noindent\makebox[3.0truein][l]{\textbf{\class, \term \/ \the\year}}
\textbf{Name:} \hrulefill \\
\noindent \makebox[3.0truein][l]{\textbf{In class work \quiz}}
\textbf{Row and Seat}:\hrulefill\\
%\vspace{0.1in}

\small
\begin{flushleft}
  \emph{
 %   “Money buys everything except love, personality, freedom, immortality, silence, peace.”}
 %   \hfill \sc{Carl Sandburg}
    “Sometimes it's a little better to travel than to arrive.”}  \hfill {\sc  Robert M. Pirsig}
 \end{flushleft}
 \large
\noindent  In class work  \quiz\/  has questions 1 through  \numquestions \/ with a total of  \numpoints\/  points.   
 This assignment is due at the end of the class period (\due).
This assignment is printed on \textbf{both} sides of the paper.
\vspace{0.1in}


\begin{questions} 

\question The human population $P$ of Long Pine, Nebraska is an exponential function of the years $T$ after the year 2000.  Specifically,
the population in the years 2000 and 2010 are given in the table

\begin{figure*}[h]
\begin{center}
\begin{tabular}{|c|c|c|} \hline
  Year & $T$  & $P$ \\ \hline
  2000 & 0    & 341  \\
  2010 & 10   & 305  \\ \hline
  \end{tabular}.
  \caption{Human population of Long Pine, Nebraska for the years 2000 and 2010.}
  \end{center}
\end{figure*}
\begin{parts}

\part [2] Find the exponential function that matches the given data.

\begin{solution}[2.0in] We just need to match to the general result in the QRS. That gives
   \begin{equation*}
       P = 341 \times \left(\frac{305}{341} \right)^{T/10}.
   \end{equation*}
   Converting this to the form $P = 341 \times \mathrm{e}^{-0.011157 \cdots \times T}$ \emph{isn't} required by the problem statement; so LIB.  And converting this to the form P = 341 \times 0.98890\cdots^T$, \emph{isn't} required by the problem statement; so again LIB.
\end{solution}

\part [2] Using your exponential function from part `a,' when will the population of Long Pine be 280?

\begin{solution}%[3.0in] 
   We need to solve $280 = 341 \times \left(\frac{305}{341} \right)^{T/10}$ for $T$.
   \begin{align*}
   \left[280 = 341 \times \left(\frac{305}{341} \right)^{T/10} \right] &=
   \left[\frac{280}{341} = \left(\frac{305}{341} \right)^{T/10} \right] &\mbox{($\div$ 341)} \\
   &= \left[\log(\frac{280}{341}) = \log \left(\left(\frac{305}{341} \right)^{T/10} \right) \right] &\mbox{(log of left and right)} \\
   &= \left[\log \left (\frac{280}{341} \right) = \frac{T}{10} \log \left(\frac{305}{341}  \right) \right] &\mbox{(log property)} \\
   &= \left[T = 10 \times \frac{\log \left (\frac{280}{341} \right)}{\log \left (\left(\frac{305}{341} \right) \right)} \right], &\mbox{(divide)} \\
   &= \left[T \approx 17.7 \mbox{years} \right]. 
   \end{align*}

\end{solution}

\end{parts}

%\hfill
%\newpage


\question [2]  Find the solution to the linear equations
\begin{align*}
   5 x + 8 y &= 14, \\
   2 x - 2 y &= 3.
\end{align*}

\begin{solution}[3.0in] Let's use our matrix method. Ordering the unknowns as $x$  (first column) and $y$ (second column), 
   our system in matrix form is
   \begin{align*}
       \begin{bmatrix} 5 & 8 & 14 \\ 2 & -2 & 3 \end{bmatrix}
       \begin{bmatrix}  \\ 2 & R_2 \leftarrow 2 R_1 - 5 R_2 \end{bmatrix}
       &= \begin{bmatrix} 5 & 8 & 14 \\ 0 & 26 & 13 \end{bmatrix}
   \end{align*}
   The second equation is $26 y = 13$. So $y=\frac{1}{2}$. Pasting that into the 
   first equation gives $5 x + 4 = 14$. So $x =2$. 

   Should we check our work?  Sure. 
   \begin{equation*}    
      \left[5 x + 8y = 14\right] = \left[5 \times 2  + 8 \times \frac{1}{2} = 14\right] =
      \left[10 + 4 = 14\right] = \mbox{True!}      
   \end{equation*}
   And one more time
   \begin{equation*}    
      \left[2 x - 2 y = 3\right] = \left[2 \times 2  - 2 \times \frac{1}{2} = 3 \right] =
      \left[4 - 1 = 3\right] = \mbox{True!}      
   \end{equation*}
\end{solution}

\question [2] Find the solution to the linear equations
\begin{align*}
   x + y + z &= 14, \\
   y - z &= 10,\\
   2 z &= 8.
\end{align*}

\begin{solution} Ha! Let's be lazy and start at the bottom and work up! That gives $z=4$;
   so $y- 4 = 10$. So $y=14$. And finally $x + 14 + 4 = 14$. So $x = -4$.

\end{solution}

\end{questions}

\end{document}
