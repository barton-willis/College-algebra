\documentclass[12pt,fleqn,answers]{exam}
\usepackage{pifont}
\usepackage{dingbat}
\usepackage{amsmath,amssymb}
\usepackage{epsfig}
\usepackage[colorlinks=true,linkcolor=black,anchorcolor=black,citecolor=black,filecolor=black,menucolor=black,runcolor=black,urlcolor=black]{hyperref}
\usepackage[letterpaper, margin=0.75in]{geometry}
\addpoints
\boxedpoints
\pointsinmargin
\pointname{pts}

\usepackage[activate={true,nocompatibility},final,tracking=true,kerning=true,factor=1100,stretch=10,shrink=10]{microtype}
\usepackage[american]{babel}
%\usepackage[T1]{fontenc}
\usepackage{fourier}
\usepackage{isomath}
\usepackage{upgreek,amsmath}
\usepackage{amssymb}

\newcommand{\dotprod}{\, {\scriptzcriptztyle
    \stackrel{\bullet}{{}}}\,}

\newcommand{\reals}{\mathbf{R}}
\newcommand{\lub}{\mathrm{lub}} 
\newcommand{\glb}{\mathrm{glb}} 
\newcommand{\complex}{\mathbf{C}}
\newcommand{\dom}{\mbox{dom}}
\newcommand{\range}{\mbox{range}}
\newcommand{\cover}{{\mathcal C}}
\newcommand{\integers}{\mathbf{Z}}
\newcommand{\vi}{\, \mathbf{i}}
\newcommand{\vj}{\, \mathbf{j}}
\newcommand{\vk}{\, \mathbf{k}}
\newcommand{\bi}{\, \mathbf{i}}
\newcommand{\bj}{\, \mathbf{j}}
\newcommand{\bk}{\, \mathbf{k}}
\newcommand{\dist}{\, \mathrm{dist}}
\DeclareMathOperator{\Arg}{\mathrm{Arg}}
\DeclareMathOperator{\Ln}{\mathrm{Ln}}
\newcommand{\imag}{\, \mathrm{i}}

\usepackage{graphicx}
\newcommand\AM{{\sc am}}
\newcommand\PM{{\sc pm}}
     
\usepackage{twemojis}
\newcommand{\quiz}{Week 1}
\newcommand{\term}{\Huge \twemoji{tulip} \normalsize}
\newcommand{\due}{9:55 \AM}
\newcommand{\class}{MATH 102}
\begin{document}
\large
\vspace{0.1in}
\noindent\makebox[3.0truein][l]{\textbf{\class}, \term \/ \the\year}
\textbf{Name:} \hrulefill \\
\noindent \makebox[3.0truein][l]{\textbf{In class work \quiz}}
\textbf{Row and Seat}:\hrulefill\\
\vspace{0.1in}


\noindent  In class work  \quiz\/  has questions 1 through  \numquestions \/ with a total of  \numpoints\/  points.   
 This assignment is due at the end of the class period (\emph{\due}).

\vspace{0.1in}


\begin{questions} 

\question [5] Find the \emph{distance} between the points $(7,9)$ and $(-1,-2)$.
\begin{solution}[2.0in]
We have
\begin{align*}
    \dist((7,9),(-1,-2)) &= \sqrt{(7+1)^2 + (9 + 2)^2}, &(\mbox{distance formula})\\
                           &= \sqrt{64+ 121}, &(\mbox{arithmetic})\\
                           &= \sqrt{185}. &(\mbox{arithmetic})
\end{align*}
The factors of 185 are $5$ and $37$.  Neither of these factors are perfect squares, so $\sqrt{185}$
is properly simplified. 
Unless asked for a decimal approximation, you should leave
your answers in an exact form. This problem \emph{doesn't} ask for an exact solution, so\
$13.60147$ is \emph{not} a correct solution.

\end{solution}

\question[5] The \emph{midpoint} of points $P$ and $(5,6)$ is $(-2,3)$. Find the 
\emph{coordinates} of the point $P$.
\begin{solution}[2.0in] Let $P = (x,y)$. We have
    \begin{equation*}
        \left(\frac{x+5}{2}, \frac{y+6}{2} \right) = (-2,3).
    \end{equation*}
    So
    \begin{align*}
        \frac{x+5}{2} &= -2,\\
        \frac{y+6}{2} &= 3.
    \end{align*}
    Solving these equations for $x$ and $y$ gives $x=-9$ and $y=0$.

\end{solution}


\question[5] Are the three points  $(7,9), (-1,-2)$, and  $(0,10)$ the vertices of
a right triangle? Explain.

\begin{solution}
We have
\begin{align*}
    \dist((7,9), (-1,2)) &= \sqrt{85} &\mbox{(problem 1)}\\
    \dist((-1,2), (0,10)) &= \sqrt{1^2 + 8^2} = \sqrt{65} \\
    \dist((0,10),(7,9)) &= \sqrt{49 + 1^2} = \sqrt{50} .
\end{align*}
The largest of these numbers is $\sqrt{85}$. But $\sqrt{85}^2 \neq \sqrt{65}^2 + \sqrt{50}^2$,
so the three points $(7,9), (-1,-2)$, and  $(0,10)$ are \emph{not} the vertices of a
right triangle.

\end{solution}

\end{questions}
\end{document}