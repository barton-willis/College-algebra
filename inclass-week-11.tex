\documentclass[12pt,fleqn,answers]{exam}
\usepackage{pifont}
\usepackage{dingbat}
\usepackage{amsmath,amssymb}
\usepackage{epsfig}
\usepackage[colorlinks=true,linkcolor=black,anchorcolor=black,citecolor=black,filecolor=black,menucolor=black,runcolor=black,urlcolor=black]{hyperref}
\usepackage[letterpaper, margin=0.75in]{geometry}
\usepackage{tikz}
\usetikzlibrary{arrows}
\addpoints
\boxedpoints
\pointsinmargin
\pointname{pts}

\usepackage{pdfpages}
\usepackage[final]{microtype}
\usepackage[american]{babel}
\usepackage[T1]{fontenc}
\usepackage{fourier}
%\usepackage{eucal}

\usepackage{isomath}
\usepackage{upgreek,amsmath}
\usepackage{amssymb}
\usepackage{siunitx}

\newcommand{\dotprod}{\, {\scriptzcriptztyle
    \stackrel{\bullet}{{}}}\,}

\newcommand{\reals}{\mathbf{R}}
\newcommand{\lub}{\mathrm{lub}} 
\newcommand{\glb}{\mathrm{glb}} 
\newcommand{\complex}{\mathbf{C}}
\newcommand{\dom}{\mbox{dom}}
\newcommand{\range}{\mbox{range}}
\newcommand{\cover}{{\mathcal C}}
\newcommand{\integers}{\mathbf{Z}}
\newcommand{\degree}{\mathrm{degree}}
\newcommand{\vi}{\, \mathbf{i}}
\newcommand{\vj}{\, \mathbf{j}}
\newcommand{\vk}{\, \mathbf{k}}
\newcommand{\bi}{\, \mathbf{i}}
\newcommand{\bj}{\, \mathbf{j}}
\newcommand{\bk}{\, \mathbf{k}}
\newcommand{\dist}{\, \mathrm{dist}}
\DeclareMathOperator{\Arg}{\mathrm{Arg}}
\DeclareMathOperator{\Ln}{\mathrm{Ln}}
\newcommand{\imag}{\, \mathrm{i}}

\usepackage{tikz}
\usepackage{amsmath}
\usetikzlibrary{arrows}
\usepackage{xcolor}
\shadedsolutions
\definecolor{SolutionColor}{rgb}{0.95,0.95,0.95}

\usepackage{graphicx}
\newcommand\AM{{\sc am}}
\newcommand\PM{{\sc pm}}
     
%\usepackage{twemojis}
\newcommand{\quiz}{11}
\newcommand{\term}{Spring}
\newcommand{\due}{9:55 \AM}
\newcommand{\class}{MATH 102}
\begin{document}
\large
\vspace{0.1in}
\noindent\makebox[3.0truein][l]{\textbf{\class, \term \/ \the\year}}
\textbf{Name:} \hrulefill \\
\noindent \makebox[3.0truein][l]{\textbf{In class work \quiz}}
\textbf{Row and Seat}:\hrulefill\\
%\vspace{0.1in}

\begin{quote}
  \emph{
    “Study hard what interests you the most in the most undisciplined, irreverent and original manner possible.” 
    \hfill \sc{Richard Feynmann}}
\end{quote}
\noindent  In class work  \quiz\/  has questions 1 through  \numquestions \/ with a total of  \numpoints\/  points.   
 This assignment is due at the end of the class period (\due).
This assignment is printed on \textbf{both} sides of the paper.
\vspace{0.1in}


\begin{questions} 

\question [2] Given that $E$ is an exponential function and that $E(0)=9$
and $E(2) = 11$, find a formula for $E$.

<<<<<<< Updated upstream
\begin{solution}[2.5in]  The formula for every exponential function $E$ has the 
    form $E(x) = C a^x$, where $C$ is the initial value and $a$ is the growth rate.
    The fact  $E(0)=9$ tells us that
=======
\begin{solution}[2.5in] The formula for $E$ has the form $E(x) = C a^x$,
    where $C$ is the initial value and $a$ is the growth rate. The 
    initial value is $E(0)$, so $C = 9.$ We have to work a bit 
    harder to find the growth rate. We have
>>>>>>> Stashed changes
    \begin{equation*}
        \left[  C a^0 = 9 \right] = \left[  C  = 9 \right]
    \end{equation*}
    So the initial value is $C=9$. And the fact $E(2) = 11$ tells us that
    \begin{equationS*}
        \left[  C a^2 = 11 \right] = \left[  9 a^2  = 11 \right] & 
            \mbox{(substitute for $C$)} \\
            
    \end{equation*}
    The parenthesis surrounding $\frac{\sqrt{11}}{3}$ in 
    $9 \left(  \frac{\sqrt{11}}{3} \right)^x$ makes it
    clear that the entire fraction is raised to the power $x$.
    Without the parenthesis, the meaning of the expression is
    unclear.

  \end{solution}

\question [2] Given that $H$ is an exponential function with
initial value of $8$ and that
\begin{equation*}
   \frac{H(4)}{H(3)} = \frac{2}{3},
\end{equation*}
find a formula for $H$.
\begin{solution}%[2.5in] 
    For any exponential function $H$, the growth rate is the quotient
    $\frac{H(x+1)}{H(x)}$, where $x$ is any real number. Specializing this 
    to $x=3$, we see that the
    growth rate is $\frac{H(3+1)}{H(3)} =  \frac{H(4)}{H(3)} = \frac{2}{3}$. Since the initial value is $8$, we
    have
    \begin{equation*}
        H(x) = 8 \times \left(\frac{2}{3} \right)^x.
    \end{equation*}

\end{solution}
%\vfill
%\newpage

\question [2]  At 6 \AM,\, Louisa has 340 mg of caffeine circulating 
in her blood. After $T$ hours, the amount of caffeine $C$ in her blood is
\(
     C = 340  \times  0.9^T
\).
When Louisa goes to bed at 10 \PM,\, how much caffeine is
still in circulation?
\begin{solution}[1.5in] We have
    \begin{equation*}
        C = 340 \times 0.9^{16} = \SI{63.0}{\milli\gram}.
    \end{equation*}
I rounded the value to the nearest tenth---given the context, that
is reasonable.
\end{solution}

\question Intense physical exercise can temporarily raise the amount
of creatinine in the blood above its normal level.\footnote{I suggest
that you \emph{not} take medical advice from a mathematician, but
if you are scheduled for a kidney function test, skipping rope
for 60 minutes followed by 20 minutes of burpees the day 
before might lead to worry and additional medical tests.} After intense exercise,
 Martin's blood creatinine level $C$ is 
\begin{equation*}
    C = 0.9 + 0.2 \times \left(\frac{1}{2} \right)^{T/4},
\end{equation*}
where $T$ is the number of hours after exercise. 
\begin{parts}
\part[2] Make a table of Martin's creatine levels after 2, 4, 8, and 16 hours.
\begin{solution}[2.5in] This problem is a calculator exercise---be sure you know how
to use your own calculator to evaluate exponential functions. 

\vspace{0.1in}
\begin{centering}
    \begin{tabular}{|c|c|} \hline
        Time (hours) &  Creatinine (mg/dL) \\ \hline \hline
        2 & 1.04 \\ \hline
        4 & 1.00 \\ \hline
        8 & 0.95 \\ \hline
        16 & 0.91 \\ \hline
    \end{tabular}
\end{centering}

    \quad I rounded these values to the nearest hundredth---given that
    the problem involved medical data, that seemed
    reasonable. The problem statement didn't include the units on the 
    creatinine---if you need to know, it's mg/dL, or milligrams per deciliter of blood.
    
\end{solution}

\part [2] Many many hours after intense exercise, what is 
Martin's blood creatinine level?  Specifically, what is the horizontal
asymptote toward infinity to the equation \(C = 0.9 + 0.2 
   \times \left(\frac{1}{2} \right)^{T/4}\)?

\end{parts}
\begin{solution} For large $T$, the term $0.2 
   \times \left(\frac{1}{2} \right)^{T/4}$ is close to zero. So
   for very large $T$, we have $C \approx 0.9$.

\end{solution}
\end{questions}
\end{document}
