\documentclass[12pt,fleqn]{exam}
\usepackage{pifont}
\usepackage{dingbat}
\usepackage{amsmath,amssymb}
\usepackage{epsfig}
\usepackage[colorlinks=true,linkcolor=black,anchorcolor=black,citecolor=black,filecolor=black,menucolor=black,runcolor=black,urlcolor=black]{hyperref}
\usepackage[letterpaper, margin=0.75in]{geometry}
\usepackage{tikz}
\usetikzlibrary{arrows}
\addpoints
\boxedpoints
\pointsinmargin
\pointname{pts}

\usepackage{pdfpages}
\usepackage[final]{microtype}
\usepackage[american]{babel}
\usepackage[T1]{fontenc}
\usepackage{fourier}
\usepackage{isomath}
\usepackage{upgreek,amsmath}
\usepackage{amssymb}

\newcommand{\dotprod}{\, {\scriptzcriptztyle
    \stackrel{\bullet}{{}}}\,}

\newcommand{\reals}{\mathbf{R}}
\newcommand{\lub}{\mathrm{lub}} 
\newcommand{\glb}{\mathrm{glb}} 
\newcommand{\complex}{\mathbf{C}}
\newcommand{\dom}{\mbox{dom}}
\newcommand{\range}{\mbox{range}}
\newcommand{\cover}{{\mathcal C}}
\newcommand{\integers}{\mathbf{Z}}
\newcommand{\degree}{\mathrm{degree}}
\newcommand{\vi}{\, \mathbf{i}}
\newcommand{\vj}{\, \mathbf{j}}
\newcommand{\vk}{\, \mathbf{k}}
\newcommand{\bi}{\, \mathbf{i}}
\newcommand{\bj}{\, \mathbf{j}}
\newcommand{\bk}{\, \mathbf{k}}
\newcommand{\dist}{\, \mathrm{dist}}
\DeclareMathOperator{\Arg}{\mathrm{Arg}}
\DeclareMathOperator{\Ln}{\mathrm{Ln}}
\newcommand{\imag}{\, \mathrm{i}}

\usepackage{xcolor}
\shadedsolutions
\definecolor{SolutionColor}{rgb}{0.95,0.95,0.95}

\usepackage{graphicx}
\newcommand\AM{{\sc am}}
\newcommand\PM{{\sc pm}}
     
%\usepackage{twemojis}
\newcommand{\quiz}{9}
\newcommand{\term}{Spring}
\newcommand{\due}{9:55 \AM}
\newcommand{\class}{MATH 102}
\begin{document}
\large
\vspace{0.1in}
\noindent\makebox[3.0truein][l]{\textbf{\class, \term \/ \the\year}}
\textbf{Name:} \hrulefill \\
\noindent \makebox[3.0truein][l]{\textbf{In class work \quiz}}
\textbf{Row and Seat}:\hrulefill\\
%\vspace{0.1in}

\begin{quote}
\emph{Mistakes are a fact of life. It is the response to the error 
that counts.} \hfill \mbox{\sc Nikki Giovanni}
\end{quote}
\noindent  In class work  \quiz\/  has questions 1 through  \numquestions \/ with a total of  \numpoints\/  points.   
 This assignment is due at the end of the class period (\due).
This assignment is printed on \textbf{both} sides of the paper.
\vspace{0.1in}


\begin{questions} 

\question Find the solution set to $\frac{2 x + 3}{4 x +1} \leq 1$ by
following these steps.

\begin{parts}

    \part [1] Use algebra tools to find an equivalent inequality of the 
    form $\frac{P(x)}{Q(x)} \leq 0$, where $P$ and $Q$ are polynomials.

    \begin{solution}[1.25in]
    
    \end{solution}

    \part[1] Find all x-intercepts and all VAs for $\frac{P(x)}{Q(x)}$.

    \begin{solution}[2.25in]
    
    \end{solution}


    \part [1] Put all x-intercepts and VAs on to a number line.

    \begin{solution}%[1.25in]
    
    \end{solution}

    \vfill 
    \newpage
    \part [1] Build the chart with columns for the interval, the test
    number, evaluation at the test number, and the true/false value.

    \begin{solution}[3.25in]
    
    \end{solution}


    \part [1] Test each interval endpoint for inclusion or
    exclusion into the solution set.

    \begin{solution}[3.25in]
    
    \end{solution}

    \part [1] Express the solution set in either interval notation, 
    pictorially, or set builder notation.

\end{parts}

\newpage

\question Find the vertex of each parabola.

\begin{parts}

    \part [1] $y -2 = 5(x+1)^2$.
    \begin{solution}[1.25in]
    
    \end{solution}

    \part [1] $y =  3 x^2 + 2 x + 9$
    \begin{solution}[1.25in]
    
    \end{solution}

    \part [1] $y =  x (1-x)$
    \begin{solution}[1.25in]
    
    \end{solution}



\end{parts}

\newpage

\question Morwenna grows and sells organic mustard greens. The number $q$ of
bunches of greens she can sell in a day is related to the selling
price of $p$ dollars per bunch by $q = 20 - 2 p$. 

\begin{parts}

    \part[1] Express the \emph{revenue} $R$ she gets for selling
    $q$ bunches of greens for $p$ dollars per bunch as a function
    of the selling price.

    \begin{solution}[2.25in]
    
    \end{solution}

    \part[1] Find the selling price $p$ that will maximize Morwenna's
    daily revenue.

\end{parts}

\vfill
\newpage

\question Sketch a pretty good graph of $y = (x-1)^2 (x+1)^2$.
\begin{solution}[3.25in]
    
\end{solution}

\question [1]  Given that $P$ is a third degree polynomial that
(a) has a zero with multiplicity of 2 at 5; (b) a zero with multiplicity 1 
at -2; and $P(0) = 1$, find an equation for $P$.
\end{questions}
\end{document}
