\documentclass[12pt,fleqn]{exam}
\usepackage{pifont}
\usepackage{dingbat}
\usepackage{amsmath,amssymb}
\usepackage{epsfig}
\usepackage[colorlinks=true,linkcolor=black,anchorcolor=black,citecolor=black,filecolor=black,menucolor=black,runcolor=black,urlcolor=black]{hyperref}
\usepackage[letterpaper, margin=0.75in]{geometry}
\usepackage{tikz}
\usetikzlibrary{arrows}
\addpoints
\boxedpoints
\pointsinmargin
\pointname{pts}

\usepackage{pdfpages}
\usepackage[final]{microtype}
\usepackage[american]{babel}
\usepackage[T1]{fontenc}
\usepackage{fourier}
%\usepackage{eucal}

\usepackage{isomath}
\usepackage{upgreek,amsmath}
\usepackage{amssymb}

\newcommand{\dotprod}{\, {\scriptzcriptztyle
    \stackrel{\bullet}{{}}}\,}

\newcommand{\reals}{\mathbf{R}}
\newcommand{\lub}{\mathrm{lub}} 
\newcommand{\glb}{\mathrm{glb}} 
\newcommand{\complex}{\mathbf{C}}
\newcommand{\dom}{\mbox{dom}}
\newcommand{\range}{\mbox{range}}
\newcommand{\cover}{{\mathcal C}}
\newcommand{\integers}{\mathbf{Z}}
\newcommand{\degree}{\mathrm{degree}}
\newcommand{\vi}{\, \mathbf{i}}
\newcommand{\vj}{\, \mathbf{j}}
\newcommand{\vk}{\, \mathbf{k}}
\newcommand{\bi}{\, \mathbf{i}}
\newcommand{\bj}{\, \mathbf{j}}
\newcommand{\bk}{\, \mathbf{k}}
\newcommand{\dist}{\, \mathrm{dist}}
\DeclareMathOperator{\Arg}{\mathrm{Arg}}
\DeclareMathOperator{\Ln}{\mathrm{Ln}}
\newcommand{\imag}{\, \mathrm{i}}

\usepackage{tikz}
\usepackage{amsmath}
\usetikzlibrary{arrows}
\usepackage{xcolor}
\shadedsolutions
\definecolor{SolutionColor}{rgb}{0.95,0.95,0.95}

\usepackage{graphicx}
\newcommand\AM{{\sc am}}
\newcommand\PM{{\sc pm}}
     
%\usepackage{twemojis}
\newcommand{\quiz}{10}
\newcommand{\term}{Spring}
\newcommand{\due}{9:55 \AM}
\newcommand{\class}{MATH 102}
\begin{document}
\large
\vspace{0.1in}
\noindent\makebox[3.0truein][l]{\textbf{\class, \term \/ \the\year}}
\textbf{Name:} \hrulefill \\
\noindent \makebox[3.0truein][l]{\textbf{In class work \quiz}}
\textbf{Row and Seat}:\hrulefill\\
%\vspace{0.1in}

\begin{quote}
  \emph{
    “I like maxims that don't encourage behavior modification.”
    \hfill \sc{Bill Watterson}}
\end{quote}
\noindent  In class work  \quiz\/  has questions 1 through  \numquestions \/ with a total of  \numpoints\/  points.   
 This assignment is due at the end of the class period (\due).
This assignment is printed on \textbf{both} sides of the paper.
\vspace{0.1in}


\begin{questions} 

\question Given that $f(x) = 2 x + 3$ and $g(x) = 1-x$, find the 
\emph{numerical value} of each of the following

\begin{parts}
\part [2] $f \circ g (4)$
\begin{solution}[1.25in]

\end{solution}

\part [2] $g \circ f (4)$
\begin{solution}[1.25in]

\end{solution}

\part [2] $g \circ g (0)$
\begin{solution}[1.25in]

\end{solution}


\end{parts}


\question [2] Given that $f(x) = 2 x + 3$ and $g(x) = 1-x$, find 
a formula for $f \circ g$.

\begin{solution}

\end{solution}
\hfill
\newpage 

\question A table of values for functions $f$ and $g$ are

\begin{tabular}[h]{|c|c|} 
  \hline
  $x$  & $f(x)$ \\ \hline \hline
  0    & 3 \\ \hline
  1    & 2 \\ \hline
  2    & 1 \\ \hline
  3    & 0 \\ \hline
\end{tabular} \phantom{xxxxxxx}
\begin{tabular}[h]{|c|c|} 
  \hline
  $x$  & $g(x)$ \\ \hline \hline
  0    & 1 \\ \hline
  1    & 3 \\ \hline
  2    & 2 \\ \hline
  3    & 0 \\ \hline
\end{tabular} 

Find the \emph{numerical values} of 
\begin{parts}
 \part[2] $ f \circ g (1)$
 \begin{solution}[1.25in]

 \end{solution}

 \part[2] $ g \circ f (1)$
 \begin{solution}[1.25in]

 \end{solution}

 \part[2] Using only the values from the table, find the solution
 to the equation \mbox{$g \circ g (x) = 2$.}

\end{parts}

\end{questions}
\end{document}
